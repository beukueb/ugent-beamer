\documentclass{beamer}
\usepackage{lipsum}

\usetheme{UniversiteitGent}
%\usetheme[framenumber,totalframenumber]{UniversiteitGent}
%\usetheme[faculty=di,framenumber,totalframenumber]{UniversiteitGent}
%\usetheme[faculty=we,usecolors,framenumber,totalframenumber]{UniversiteitGent}
%\usetheme[faculty=lw,language=english,framenumber,totalframenumber]{UniversiteitGent}

\title{Some slides with a UGent \texttt{beamer} theme}
\subtitle{This is a dummy subtitle}
\author{Pieter Belmans}

\begin{document}

\begin{frame}
  \titlepage
\end{frame}

\section{My section}
\subsection{My subsection}

\begin{frame}
  \frametitle{A first test frame}

  \lipsum[2]
\end{frame}

\begin{frame}
  \frametitle{Test frame with overflow}

  \lipsum[2-3]
\end{frame}

\begin{frame}
  \frametitle{The is a test frame with a pretty long frame title}

  \lipsum[2]
\end{frame}


\subsection{My subsection 2}
\begin{frame}
  \frametitle{Test frame with itemize}

  \begin{itemize}
    \item<1-> firstly
    \item<2-> secondly
      \begin{itemize}
        \item sub-item
        \item another sub-item
      \end{itemize}
    \item<3-> thirdly
  \end{itemize}
\end{frame}

\begin{frame}
  \frametitle{A math frame}
  
  \begin{theorem}[Pythagoras]
    The square of the hypotenuse of a \alert{right} triangle is equal to the sum of the squares on the other two sides:
    \[
      a^2 + b^2 = c^2.
    \]
  \end{theorem}
  \begin{proof}
    Straightforward.
  \end{proof}
\end{frame}

\begin{frame}
  \frametitle{Environments}
  
  \begin{definition}
    A \textbf{prime number} (or a prime) is a natural number which has exactly two distinct natural number divisors: 1 and itself. 
  \end{definition}
  
  \begin{exampleblock}{Example}
    The first five prime numbers are $2$, $3$, $5$, $7$, and $11$.
  \end{exampleblock}
  
  \begin{alertblock}{Alert block}
    Note that $1$ is not a prime number.
  \end{alertblock}
\end{frame}

\end{document}
