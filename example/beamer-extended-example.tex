%This is an extended example with as many different features for presentations

\documentclass{beamer}
%\documentclass[12pt,handout]{beamer}
\usepackage[latin1]{inputenc}

%Useful packages [comment out what is not needed]
\usepackage[overlay,absolute]{textpos} %for positioning graphics absolutely
\usepackage{soul} %for crossing out text with \st{}
\usepackage{multimedia} %for animations
\usepackage{listings} %for code
\usepackage{etoolbox} %for using settings (if...)

%Settings
% can either be set here, or given as commandline argument
% Example setting as command line argument:
%  pdflatex "\newcommand\presentationPlatform{linux}\input{presentation}"
\providecommand{\presentationPlatform}{linux} %linux, win or mac

%Theme
%%Structure
\usetheme{AnnArbor}%AnnArbor,Warsaw,Madrid,Darmstadt
%\useoutertheme{infolines}
%\useinnertheme{rectangles}
\usetheme[faculty=fw]{MiniUGent}

%%Colors
%\usecolortheme{wolverine}
%\usecolortheme[faculty=fw]{UniversiteitGent}
%\usecolortheme{UniversiteitGent}

%Development settings
%\setbeamertemplate{background}[grid][step=1cm] %useful for positioning

%%% Preamble
\title[Forensic deep sequencing] 
{Porting Forensic DNA Analysis To Deep Sequencing}
%\subtitle{My-Forensic-Loci-queries and the Forensic-Loci-Allele-Database}
\author[C. Van Neste]
{ir.~Christophe~Van~Neste} %\and S.~Anders\inst{2}}
\institute[LabFBT] % (optional)
{
  %\inst{1}%
  Laboratory of Pharmaceutical Biotechnology\\
  University Ghent
%  \and
%  \inst{2}%
%  Institute of Theoretical Philosophy\\
%  University There
}
\date[June 11, 2015] % (optional)
{PhD defense, 2015}
\subject{Bioinformatics}

\begin{document}
%Titlepage
 \frame[t,plain]{ %align from top and plain
   %\mode<presentation>{\includegraphics[width=\textwidth]{images/ugent_FW_logobalk.pdf}}
   \titlepage
 }
 
 %TOC
 \begin{frame}
   \frametitle{Table of Contents}
   \tableofcontents
  \end{frame}
 
%For long pubblications use parts (>30min)
% \part{Forensic DNA analysis}
% \partpage

\section{Introduction: dissecting the title}
\begin{frame}
   \transwipe[direction=90]<1>
   \transduration<2-3>{0.4} %Example "stop motion" presentation video
  \begin{center}
    {\large Introduction: dissecting the title\\}
    \vspace{0.5cm}
    \pgfimage[height=2cm]{images/veniceport}
    \pgfimage[height=2cm]{images/crime-scene}
    \pgfimage[height=2cm]{images/deep-sea-exploration}
    \pgfimage[height=2cm]{images/quenching}\\
    \begin{columns}
      \tiny
      \column{0.17\textwidth}
      \column{0.23\textwidth}
      \uncover<2->{\alert{Port}ing}
      \column{0.3\textwidth}
      \uncover<3->{\alert{forensic DNA analysis}}
      \column{0.3\textwidth}
      \uncover<4->{to \alert{deep se}\st{a}\alert{quenc}\st{h}\alert{ing}}
    \end{columns}
  \end{center}
\end{frame}

 \subsection{Forensic DNA analysis}
 \begin{frame}[t]{Forensic DNA analysis}
   \ifdefstring{\presentationPlatform}{linux}
   {\transdissolve<3-4>[duration=0.5]} %for linux
   {\transfade<3-4>} %Works only on Acrobat
   \begin{itemize}
     \item<1-> Crime scene investigation (CSI)
       \only<2>{
         \vspace{1cm}\\
         \pgfimage[height=4cm,width=7cm]{images/csi}
       }
       \only<3>{
         \vspace{1cm}\\
         \pgfimage[height=4cm,width=7cm]{images/csimiami}
       }
       \only<4>{
         \vspace{1cm}\\
         \pgfimage[height=4cm,width=7cm]{images/csigent1}
       }
       \only<5>{
         \vspace{1cm}\\
         \pgfimage[height=4cm,width=7cm]{images/csigent2}
       }

     \item<6-> DNA identification of individuals
       \begin{itemize}
       \item<7-> Connecting individuals to places or objects
       \item<8-> Kinship
       \end{itemize}
     \item<9-> Invented 30 years ago by Alec Jeffreys
     \item<10-> Tandem repeat length-based DNA profiling

     \only<9->{
       \begin{textblock}{4}(10.8,4)
         \pgfimage[height=4cm]{images/Alec_Jeffreys}
       \end{textblock}
     }
   \end{itemize}
 \end{frame}

%Example section without logo
{\setbeamertemplate{logo}{}
 \begin{frame}[t]
   \frametitle{Without logo frame 1}
 \end{frame}
 \frame{Without logo frame 2}
} %end no logo frames

%Example portable multimedia
 \begin{frame}[t]
   \frametitle{DNA: what, where, why, and how}
   \ifdefstring{\presentationPlatform}{linux}
   {
     \movie[autostart,showcontrols]{\pgfimage[width=0.5\textwidth]{images/pcrcover}}{images/pcr.ogv}\\ %linux
   }
   {
     \movie[autostart,showcontrols]{\pgfimage[width=0.5\textwidth]{images/pcrcover}}{images/pcr.mov}\\ %mac+win
   }
   % \movie[height=4cm,width=6cm,showcontrols,poster,borderwidth=0.1cm]... 
   %General link for backup solution (will be opened externally)
   \movie[externalviewer]{\tiny DNA amplification}{images/pcr.mp4} %general
   
%Commands
%drop sound and speed up: ffmpeg -i pcr.mp4 -an -filter:v "setpts=0.5*PTS" pcr_ws.mp4
%encoding => ffmpeg -i pcr_ws.mp4 pcr.mov
%            ffmpeg -i pcr_ws.mp4 -qscale:v 7 pcr.ogv
%cover    => ffmpeg -ss 00:01:12 -i pcr.mp4 -vframes 1 -q:v 2 pcrcover.jpg
%%Reduce size: ffmpeg -i MiSeqFGx.mp4 -c:a copy -vf scale=640:-1 MiSeqFGx.mov
%%%%%%%%%%%%%%%%%%%%%%%%%
%Useful commands for making screenvideo
%xwininfo #xy positions window for screenvideo - alternative: xdotool getmouselocation
%xdotool search --name "MyFLsite - Chromium" windowsize 1500 1000
%ffmpeg -video_size 1500x1000 -framerate 25 -f x11grab -i :0.0+78,21  resultMyFLq.mp4 #first record as mp4 for good quality
%ffmpeg -i resultMyFLq.mp4 -qscale:v 7 resultMyFLq.ogv #Standard video ogv quality is very low. qscale 1-10, 5-7 usually good
%ffmpeg -i resultMyFLq.mp4 -pix_fmt yuv420p resultMyFLq.mov #Specific pixel format needed for mov to play on mac/windows!
 \end{frame}

\subsection{Copy paste examples}
%Column example
 \begin{frame}
   \frametitle{Columns}
   \begin{columns}
     \column{0.6\textwidth}
     0.6 of the width
     \column{0.4\textwidth}
     \pause
     Smaller columns, shown in second slide\\
   \end{columns}
 \end{frame}

%Item example
%For items it is often useful to have top alignment
 \begin{frame}[t]
   \frametitle{Items}
   \begin{itemize}[<+->]
   \item Generic processing: not only STRs
   \item No full homopolymer compression
   \item Automatic determination of flanks
   \end{itemize}
 \end{frame}

\end{document}

%Tips
%%%%%
% \uncover<x-> usually better than \only<x->
% \only<x> usually better than \uncover<x> unless it is the last slide
%
%Undocumented, but possible transitions
% \transpush \transfade \transuncover \transcover


